\documentclass[../main.tex]{subfiles}
\begin{document}

Die vorgestellten Ergebnisse sind unter einigen Einschränkungen zu betrachten.
Als erstes ist hier aufzulisten, dass die theoretisch erarbeiten Inhalte und daraus resultierenden Hypothesen nur ein Teil aller möglichen Herausforderungen abdecken.
Daraus resultierenden ist die Umfrage so gestaltet, dass Sie nur den Teil aller Herausforderungen abgefragt werden.
Eventuell fehlende Aspekte, die allerdings eventuell unerwartet wichtig sind, wurde dementsprechend vernachlässigt.

-no clear midpoint defined in survey
-no subjective/personal situation of consultants taking into account
-number of 5-point is contreversially discussed in literature (all other numbers of points too)
-can arithmetic operations be used (to compute average and box plot)?
-no completely clear anchors are set (is it fully agree or often happens) --> loads of room for subjectivnss
-quality of the statements is to be discussed
\autocite{tanujaya2022likert}

außerdem mögliche aspekte auf die ich noch eingehen könnte:
- farb design: lila, welche wirkung hat die farbe?
- anzahl an umfrage teilnehmener
- subjektivität der teilnehmenden
- anonymität? keine genaueren infos über die teilnehmenden außer dass Sie mitarbeitende des A\&I consulting bei SAP sind
- umfrage nur in SAP und nicht außerhalb
\end{document}