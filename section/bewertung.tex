\documentclass[../main.tex]{subfiles}
\begin{document}

Die erarbeiteten Ergebnisse sind unter einigen Einschränkungen zu betrachten. 
Die theoretische Betrachtung der Kommunikationsherausforderungen bildet nur einen Ausschnitt möglicher Probleme in der Kundenkommunikation ab. 
Die aus der Theorie erarbeiteten Umfrageaussagen sind so gestaltet, dass sie nur einen Teil aller Herausforderungen abdecken und es gibt eventuell fehlende Aspekte, deren Bedeutung falsch eingeschätzt wurde und die vernachlässigt wurden. 
Des Weiteren sind die Aussagen möglicherweise nicht klar formuliert worden. 
Dies wurde in einem Fall im Freifeld bemängelt.

Auch die Likert-Skala ist kritisch zu betrachten. 
Sie definiert nur die Bewertungen 1 und 5 klar, während 2, 3 und 4 interpretierbar bleiben. 
Damit ist insbesondere unklar, was der Mittelpunkt 3 bedeutet. 
Möglichkeiten wären ein „passiert manchmal“ oder auch ein „neutral“. 
Diese Differenzierung wurde allerdings nicht vorgenommen. 
Zudem wird auch die Anzahl an Auswahlmöglichkeiten, hier 5, in der Literatur kontrovers diskutiert und 7-, 9- oder 11-Punkt-Likert Skalen werden teilweise als geeigneter betrachtet. 
Auch das Rechnen mit den Ergebnissen der Likert-Skala wird hinterfragt, z. B. ob das Ermitteln der Durchschnittswerte valide ist. 
Außerdem ist unklar, ob der qualitative Abstand zwischen den unterschiedlichen Antwortkategorien gleich ist, also ob z. B. der Unterschied von Kategorie 4 zu 5 der gleiche ist wie von 3 zu 4.
\autocite{tanujaya2022likert}

Zuletzt sollte hinterfragt werden, wie repräsentativ die Auswertung der Umfrage ist.
Die Zahl der Teilnehmenden ist gering und die Umfrage beschränkt sich ausschließlich auf die Abteilung „Analytics \& Insights Consulting“ bei SAP. 
Über die Teilnehmenden ist nichts Näheres bekannt, z. B. die Dauer der Betriebs- oder Abteilungszugehörigkeit. 
Außerdem wurde die Kundenseite nicht befragt und die Ergebnisse spiegeln nur eine Seite der Kommunikation wider. 
Dementsprechend ist das Ergebnis dieser Arbeit nur eingeschränkt repräsentativ.

\end{document}