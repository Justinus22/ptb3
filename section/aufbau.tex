\documentclass[../main.tex]{subfiles}
\begin{document}

Tabelle \ref{tab:aussagen} zeigt die zu bewertenden Aussagen der Umfrage, sowie deren durchschnittliche erreichte Punktzahl.
Die Teilnehmenden wurden instruiert, die Aussagen auf einer symmetrischen 5-Punkt-Likert-Skala von 1 “stimme nicht zu” beziehungsweise “passiert nie” bis 5 “stimme zu” beziehungsweise “passiert häufig” (in der Umfrage in Englisch) zu bewerten \autocite{joshi2015likert}. 
Außerdem gab es die Möglichkeit, in einem Freitextfeld weitere Aspekte hinzuzufügen.
Als Plattform dafür wurde Google Forms im voreingestellten Design genutzt.

\begin{dontcount}
    \small
    \spaceskip=0.2em plus 1em minus 0.1em
    \renewcommand{\arraystretch}{0.85}
    \begin{xltabular}[h]{\linewidth}{| N X | m{2.2cm}|}
        \caption{Aussagen zur Kundenkommunikation mit Durchschnittsergebnis \label{tab:aussagen}} \\
        \hline
        \multicolumn{1}{|c}{} & Aussage & Durchschnitt \\
        \hline
        \hline
        \label{q:1} &\hspace{-0.8em}Frequently changing customer requirements represent a challenge to analytics consulting.& 4,2\\
        \hline  
        \label{q:2}&\hspace{-0.8em}Unrealistic customer requirements represent a challenge to analytics consulting. & 3,87 \\
        \hline
        \label{q:3}&\hspace{-0.8em}Consultants misinterpreting customer requirements represent a challenge to analytics consulting. & 3,07  \\
        \hline
        \label{q:4}&\hspace{-0.8em}Customers having trouble identifying and/or articulating their requirements represents a challenge to analytics consulting.  & 4,07  \\
        \hline
        \label{q:5}&\hspace{-0.8em}A lack of customer motivation represents a challenge to analytics consulting. & 3,8  \\
        \hline
        \label{q:6}&\hspace{-0.8em}Consultants misunderstanding the customer during communication represents a challenge to analytics consulting. & 3,87  \\
        \hline
        \label{q:7}&\hspace{-0.8em}Customers misunderstanding the consultants during communication represents a challenge to analytics consulting. & 3,73  \\
        \hline
        \label{q:8}&\hspace{-0.8em}Not having clear goals for a project represents a challenge to analytics consulting. & 4,53  \\
        \hline
        \label{q:9}&\hspace{-0.8em}Distrust between customer and consultants represents a challenge to analytics consulting. & 3,73  \\
        \hline
        \label{q:10}&\hspace{-0.8em}A lack of respect between customers and consultants represents a challenge to analytics consulting. & 3,47  \\
        \hline
        \label{q:11}&\hspace{-0.8em}Consultants setting too high expectations represents a challenge to analytics consulting. & 2,79  \\
        \hline
        \label{q:12}&\hspace{-0.8em}Language barriers between customers and consultants represent a challenge to analytics consulting. & 2,67  \\
        \hline
        \label{q:13}&\hspace{-0.8em}Not frequent enough customer communication represents a challenge to analytics consulting. & 3,6  \\
        \hline
        \label{q:14}&\hspace{-0.8em}A lack of customer feedback represents a challenge to analytics consulting. & 3,73  \\
        \hline
        \label{q:15}&\hspace{-0.8em}A lack of a suitable communication system between consultant and customer represents a challenge to analytics consulting. & 3,2  \\
        \hline
    \end{xltabular}
\end{dontcount}
\end{document}
