\documentclass[../main.tex]{subfiles}
\begin{document}

Herausforderungen in der Kommunikation gibt es in unterschiedlichen Ausprägungen.
Genannt seien hier Beispiele wie zwischenmenschliche Frustration oder Streit, möglicherweise resultierend aus Fehlern im Projekt, wie z. B. ein Kunde, der eine Frage falsch beantwortet oder ein Consultant, der ein technisches Projektmerkmal falsch umsetzt \autocite{suleiman2022causes}.
Um die Komplexität solcher Herausforderungen herunter zu brechen, dient Luhmanns Unterteilung in die wesentlichen Bestandteile der Kommunikation hier als Grundlage.
Im Folgenden werden Luhmanns drei Kernteile der Kommunikation näher ausgeführt, indem zunächst die Herausforderungen im Informationsteil der Kommunikation, also die Anforderungen an ein Projekt, genauer beschrieben werden.
Im zweiten Schritt wird der Ausdruck betrachtet und zuletzt typische Verständnisfragen beleuchtet.
Die Herausforderungen aus den drei Kategorien sind allerdings in ihrer Ursache und Wirkung nicht als alleinstehend zu betrachten, sondern als fließend, überlappend und sich gegenseitig bedingend.

\end{document}