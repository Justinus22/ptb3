\documentclass[../main.tex]{subfiles}
\begin{document}

In Abbildung \ref{fig:Kastengrafik} ist eine Kastengrafik zu sehen, die im folgenden zur Analyse der Umfrageergebnisse genutzt wird.
Die Grafik ist nach Kastengrafik-Konvention aufgebaut und ermöglicht einen Einblick in die Verteilung der Bewertungsergebnisse \autocite{williamson1989box} berechnet durch pgfplots \autocite{feuer2015manal}.
Der Bewertungsdurchschnitt wird zusätzlich in rot gezeichnet angezeigt.
\begin{figure}[H]
\centering
\label{fig:Kastengrafik}
\begin{tikzpicture}
    \begin{axis}[
        boxplot/draw direction=y,
        x=0.82cm,
        ytick={1,...,5},
        xtick={1,...,15},
        xticklabel=\pgfmathprintnumber{\tick},
        axis x line*=bottom,
        axis y line*=left,
        xlabel={Fragen},
        ylabel={Bewertung},
        enlarge x limits,
        boxplot/average={auto},
        boxplot/every average/.style={mark=diamond*},
        /pgfplots/boxplot/hide outliers,
        fill=red
        ]
        \addplot [boxplot] table[y index=0] {data.dat};
        \addplot [boxplot] table[y index=1] {data.dat};
        \addplot [boxplot] table[y index=2] {data.dat};
        \addplot [boxplot] table[y index=3] {data.dat};
        \addplot [boxplot] table[y index=4] {data.dat};
        \addplot [boxplot] table[y index=5] {data.dat};
        \addplot [boxplot] table[y index=6] {data.dat};
        \addplot [boxplot] table[y index=7] {data.dat};
        \addplot [boxplot] table[y index=8] {data.dat};
        \addplot [boxplot] table[y index=9] {data.dat};
        \addplot [boxplot] table[y index=10] {data.dat};
        \addplot [boxplot] table[y index=11] {data.dat};
        \addplot [boxplot] table[y index=12] {data.dat};
        \addplot [boxplot] table[y index=13] {data.dat};
        \addplot [boxplot] table[y index=14] {data.dat};
    \end{axis}
\end{tikzpicture}
\caption{Kastengrafik zur Antwortverteilung}
\end{figure}

Die drei höchst bewertesteten Aussagen sind \ref{q:1}, \ref{q:4} und \ref{q:8}.
Alle drei Aussagen lassen sich in den Bereich \nameref{subsec:anforderungen} einteilen und es lässt sich vermuten, dass dementsprechend die größten Herausforderungen der Kundenkommunikation durch Auswahl von Information entsteht.

% \begin{figure}[H]
% \centering
% \label{fig:bereichdurchschnitt}
%     \begin{tikzpicture}
%         \begin{axis}[
%            ybar,
%            symbolic x coords={a,Anforderungen, Verständnis, Umgebungsbedingungen,z},
%            bar width={1.5cm},
%            width={10cm},
%            height={7cm},
%            ylabel={Bewertung},
%            x tick label style={anchor={west},rotate=15},
%            xticklabel pos=upper,
%            xtick=data,
%            ytick={1,...,5},
%            ymax=5,
%            ymin=0,
%            xmin={a},
%            xmax={z},
%            nodes near coords,
%         ]
%        \addplot[fill=cyan, draw=cyan!60!black] coordinates {
%         (Anforderungen,3.8) (Verständnis,3.422222222) (Umgebungsbedingungen,3.613445378)
%         };
%         \end{axis}
%     \end{tikzpicture}
% \caption{durchschnittliche Bewertung nach vorgestellten Bereichen}
% \end{figure}


\end{document}


