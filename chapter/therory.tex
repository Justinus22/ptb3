\documentclass[../main.tex]{subfiles}
\begin{document}

Nach \textcite{luhmann1992communication} ergibt sich Kommunikation aus drei miteinander verflochtenen Teilen: der Auswahl von Information, der Auswahl von Ausdruck und dem Verständnis oder Missverständnis dieser Informationen und des Ausdrucks.
Damit die Kommunikation erfolgreich ist, also Verständnis erzeugt, ist die adäquate Auswahl von Informationen und Ausdruck notwendig.
Alle kommunikativen Herausforderungen lassen sich demnach auf Missverständnisse durch eine unangepasste Auswahl zurückführen.

Ein Modell für die Kommunikation zwischen Kunden und Consulting bieten \textcite{davis2006communication} in Abbildung \ref{fig:kenntnisbereich}.
Dort gibt es einen Kenntnisbereich (a), in dem sich beide Parteien vorhandenes Wissen teilen.
Die kritischen Bereiche sind (b) und (c), da dort das Potenzial für den Bereich (d) generiert wird.
\textcolor{blue}{Die kritischen Bereiche sind (b) und (c), da dort Potenzial besteht über Kommunikation den Bereich (a) zu erweitern (und Wissen auszutauschen, um das Projekt nach den tatsächlichen Vorstellungen des Kunden umzusetzen).
Der Bereich (d) bietet die Möglichkeit im Verlauf des Projekts noch weitere Erkenntnisse zu gewinnen, die eine neue Erfahrung für beide Parteien bietet.}
Die Kommunikation durch Auswahl im Sinne von Luhmann entscheidet über die Größe dieses Bereichs.
\textcolor{blue}{Die Kommunikation durch Auswahl im Sinne von Luhmann verändert über die Größe dieses Bereichs.}
\begin{figure}[H]
    \centering
    \includegraphics[scale=.6]{bilder/verständnisraum.png}
    \caption{Kenntnisbereiche in der Kunden-Consultant-Beziehung (eigene Darstellung)}
    \label{fig:kenntnisbereich}
\end{figure}

\end{document}