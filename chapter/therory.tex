\documentclass[../main.tex]{subfiles}
\begin{document}

Nach \textcite{luhmann1992communication} ergibt sich Kommunikation aus drei mit einander verflochtenen Teilen: der Auswahl von Information, der Auswahl von Ausdruck und dem Verständnis oder Missverständnis dieser Informationen und des Ausdrucks.
Damit die Kommunikation erfolgreich ist, also Verständnis erzeugt, ist dementsprechend adequate Auswahl von Information und Ausdruck notwendig.
Alle kommunikativen Herausforderungen in dieser Arbeit lassen sich auf ein Missverständnis durch unangepasste Auswahl zurückführen.

Ein Modell für die Kommunikation zwischen Kunden und Consulting bieten \textcite{davis2006communication} in Abbildung \ref{fig:kenntnisbereich}.
Dort gibt es einen Kenntnisbereich (a) in dem sich beide Parteien Wissen teilen.
Kommunikation ist vor allem für die Bereichen (b) und (c) wichtig, da dort das Potenzial besteht nur durch die Auswahl.
\begin{figure}[ht]
    \centering
    \includegraphics[scale=.6]{"bilder/verständnisraum.png"}
    \caption{Kenntnisbereiche in der Kunden-Consultant-Beziehung (eigene Darstellung)}
    \label{fig:kenntnisbereich}
\end{figure}

\end{document}