\documentclass[../main.tex]{subfiles}
\begin{document}

Nach \textcite{luhmann1992communication} ergibt sich Kommunikation aus drei miteinander verflochtenen Teilen: der Auswahl von Information, der Auswahl von Ausdruck und dem Verständnis oder Missverständnis dieser Informationen und des Ausdrucks.
Damit die Kommunikation erfolgreich ist, also Verständnis erzeugt, ist die adäquate Auswahl von Informationen und Ausdruck notwendig.
Alle kommunikativen Herausforderungen lassen sich demnach auf Missverständnisse durch eine unangepasste Auswahl zurückführen.

Ein Modell zur Veranschaulichung der Kommunikation zwischen Kunden und Consulting bieten \textcite{davis2006communication} in Abbildung \ref{fig:kenntnisbereich}.
Dort gibt es einen Kenntnisbereich (a), in dem sich beide Parteien vorhandenes Wissen teilen.
Die kritischen Bereiche sind (b) und (c), da dort das Wissen nur je einer der beiden Parteien vorhanden ist.
Nur die gelungene Kommunikation, d. h. erfolgreicher Wissensaustausch mit Verständnis kann den Bereich (a) vergrößern. 
Im Bereich (d) liegen gänzlich unbekannte Informationen. 
Er bietet die Möglichkeit, im Verlauf des Projekts durch Bewusstwerdung noch weitere Erkenntnisse zu gewinnen, die beiden Parteien weiteren Wissensaustausch ermöglichen können. 

\begin{figure}[H]
    \centering
    \includegraphics[scale=.6]{bilder/verständnisraum.png}
    \caption{Kenntnisbereiche in der Kunden-Consultant-Beziehung (eigene Darstellung)}
    \label{fig:kenntnisbereich}
\end{figure}

Für die theoretische Betrachtung typischer Probleme in Kundenbeziehungen dient Luhmanns Unterteilung in die drei wesentlichen Bestandteile der Kommunikation hier als Grundlage.
Im Folgenden werden Herausforderungen in der Kommunikation zwischen Kunden und Consultants näher betrachtet. 
Zunächst wird der Informationsteil, also die Beschreibung von Anforderungen an ein Projekt untersucht. 
Im zweiten Schritt wird der Ausdruck und zuletzt die Rolle des Verständnisses betrachtet. 
Die Herausforderungen aus den drei Kategorien sind in ihrer Ursache und Wirkung nicht als alleinstehend zu betrachten, sondern als fließend, überlappend und sich gegenseitig bedingend.

\end{document}