\documentclass[../main.tex]{subfiles}
\begin{document}

Daten haben einen entscheidenden Wert für Unternehmen, da sie den Gewinn von Wissen ermöglichen.
Wissen wiederum kann bei strategischen, taktischen oder organisatorischen Entscheidungen eingesetzt werden, also fundiertere Schlüsse möglich machen.
Um den Wissensvorteil zu realisieren, müssen die Daten also nutzbar gemacht werden. Dafür ist Analytics erforderlich.
\autocite{monino2021data, gupta2020digital, sarikaya2018we}

Zusammengefasst beschreibt der Begriff Analytics das Sammeln, Gewichten, Auswählen, Vorverarbeiten, Transformieren sowie Interpretieren von Daten \autocite{tsai2015big}.
Oft ist das Ergebnis dieser Analytics-Arbeit ein Dashboard, also eine Daten-visualisierende Ansicht, die Diagramme und Grafiken sowie funktionale Elemente (zur Konfiguration des Angezeigten) enthalten kann \autocite{sarikaya2018we}.
Auf Grundlage dieser Anzeige wird es den Nutzenden des Dashboards ermöglicht, schnell Wissen zu gewinnen, das für unternehmerische Entscheidungen von zentraler Bedeutung ist \autocite{sarikaya2018we}.

Obwohl es heutzutage vielen Unternehmen rein technisch möglich wäre Analytics zu nutzen, fehlt es an Kompetenz, um die eigenen Daten effektiv zu verwerten.
Da die Unternehmen allerdings auch nicht auf den Wert ihrer Daten verzichten wollen, wird häufig externe Beratung im Bereich Analytics in Anspruch genommen.
\autocite{gupta2020achieving,chen2012business}

Neben rein technischen Problemen in diesem Beratungsprozess resultiert die Interaktion und Kommunikation mit dem Kunden in Herausforderungen für beide Parteien.
Dass diese Kommunikation allerdings optimal abläuft, ist ein kritischer Erfolgsfaktor in dem Beratungsprojekt und für die Kundenbeziehung.
\autocite{appelbaum2005critical}

Die kommunikativen Herausforderungen im Analytics Bereich wird dieser PTB nach einer theoretischen Erarbeitung typischer Schwierigkeiten durch eine Umfrage aus der Praxis ergänzen und bewerten.

\end{document}