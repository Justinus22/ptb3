\documentclass[../main.tex]{subfiles}
\begin{document}

Täglich entstehen in jedem Unternehmen weltweit neue Daten.
Die schnell wachsende Menge dieser Daten kann sie unüberschaubar machen.
Sie enthalten jedoch wertvolle mittel- und unmittelbare Informationen für Unternehmen, die es herauszufiltern und nutzbar zu machen gilt.
Um diese Werte zu heben, gibt es die Unternehmensanalytik, dich sich unter anderem mit dem Sortieren, Filtern, Transformieren und Anzeigen von Daten beschäftigt, sodass sie letztlich für Menschen übersichtlich dargestellt werden können.
Um diesen Prozess zu ermöglichen, gibt es Analytik-Softwareanwendungen, etwa von Microsoft, Google oder SAP.\@
Da die Arbeit mit solcher Software jedoch bereits auf der technischen Ebene komplex ist, lagern Unternehmen diese Aufgabe aus und wenden sich an das Analytics Consulting, holen sich also Beratung für die Analytik.
Für eine gelungene Beratung ist der Kommunikationsprozess zwischen Unternehmen und Consultingfirma entscheidend und die Vermeidung von Missverständnissen nicht selten eine Herausforderung.
 
Diese Arbeit wird untersuchen, wo diese kommunikativen Herausforderungen im Analytics Consulting liegen.
Dafür wird zunächst theoretisch erarbeitet, welche Probleme es in der Kundenkommunikation geben kann. Die drei erarbeiteten Schwerpunktbereiche werden anschließend näher untersucht:~\nameref{subsec:verstaendins}, ~\nameref{subsec:anforderungen} und~\nameref{subsec:umgebung}.
\\
Nach dieser theoretischen Betrachtung wird eine Umfrage aus der Praxis von Analytics-Beratern vorgestellt und ausgewertet, sodass schließlich ein Fazit für die häufigsten Schwierigkeiten in der Kundenkommunikation gezogen werden kann.
\end{document}