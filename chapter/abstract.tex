\documentclass[../main.tex]{subfiles}
\begin{document}

Täglich entstehen in jedem Unternehmen weltweit neue Daten.
Die schnell wachsende Menge dieser Daten kann sie unüberschaubar machen. Sie enthalten jedoch wertvolle mittel- und unmittelbare Informationen für Unternehmen, die es herauszufiltern und nutzbar zu machen gilt.
Um diese Werte zu heben, gibt es die Unternehmensanalytik, dich sich unter anderem mit dem Sortieren, Filtern, Transformieren und Anzeigen von Daten beschäftigt, sodass sie letztlich für Menschen übersichtlich dargestellt werden können.
Um diesen Prozess zu ermöglichen, gibt es Analytik-Softwareanwendungen, wie etwa von Microsoft, Google oder SAP.
Da die Arbeit mit solcher Software jedoch bereits auf der technischen Ebene komplex ist, lagern Unternehmen diese Aufgabe aus und wenden sich an das Analytics Consulting, holen sich also Beratung für die Analytik.
Für eine gelungene Beratung ist dabei der Kommunikationsprozess zwischen Unternehmen und Consultingfirma entscheidend und die Vermeidung von Missverständnissen nicht selten eine Herausforderung.
 
Diese Arbeit wird untersuchen, wo diese Herausforderungen liegen.
Dafür wird zuerst theoretisch erarbeitet, welche Probleme es in der externen Kommunikation, von der Beratung zum Kunden beziehungsweise von dem Kunden zur Beratung, und in der internen Kommunikation, innerhalb der Beratung und innerhalb des Kundenunternehmen, gibt. 
\\
Nach dieser Betrachtung wird ein Umfrage von Analytik-Beratern vorgestellt, die die Probleme gewichtet und für die Analytik einordnet.
\end{document}