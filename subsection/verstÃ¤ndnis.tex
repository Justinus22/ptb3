\documentclass[../main.tex]{subfiles}
\begin{document}

Herausforderungen des Verständnisses beziehungsweise des Wissens sind diejenigen, bei dem der Kunde oder der Consultant etwas nicht weiß oder nicht kann, das relevant für die Kommunikation ist.
Das heißt, dass die Auswahl von Information und Ausdruck ungenügend für ein Verständnis war.
\textcolor{red}{Dies beinhaltet das Kenntnis des Consultant von der Branche des Kunden und denjenigen Aspekten, die für den Kunden selbstverständlich erscheinen, aber unklar für den Consultant sind \autocite{appelbaum2005critical, gamil2017identification, davis2006communication, sutter2019consultants}.
Dafür muss allerdings auch der Kunde eine Vorstellung von seinen eigenen Bedürfnissen haben, was oft nicht gegeben ist \autocite{davis2006communication}.}

Des weiteren ergeben sich Herausforderungen daraus, dass Kunden und Beratung unterschiedliche Denkweise und Perspektive haben und sich dadurch verkennen \autocite{sutter2019consultants}.
Auch Sprachbarrieren oder Ungenauigkeiten von natürlicher Sprache können hinderlich sein \autocite{sayer2013misunderstanding,gamil2017identification}.

Dieses gegenseitige (Miss-) Verständnis behindert allerdings nicht nur die Kommunikation sondern auch den Projekterfolg an sich und ist deshalb mehrfach relevant \autocite{appelbaum2005critical}.

\end{document}