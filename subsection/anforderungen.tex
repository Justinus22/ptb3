\documentclass[../main.tex]{subfiles}
\begin{document}

Anforderungen sind ausschlaggebend für das Verständnis eines Consultants über ein Projekt und sind damit ein relevanter Aspekt der Kundenkommunikation \autocite{davis2006communication}.
Das eindeutige Identifizieren und Artikulieren von den Anforderungen ist dabei ein Schlüsselpunkt, welcher wiederum auf Verständnis von Kundenseite zurückzuführen ist \autocite{gamil2017identification, chakrabarti2004identification, salado2021systems}.

Deshalb können Anforderungen initial unverständlich formuliert oder gar nicht den eigentlichen Bedürfnissen entsprechend vorliegen und es entstehen Missverständnisse \autocite{bjarnason2017role}.
Demnach werden die Anforderungen oft geändert, ein unvermeidlicher und für den Projekterfolg wichtiger Prozess, der allerdings auch die kommunikative Herausforderungen mit sich bringt, dass sich der Consultant immer wieder neu auf den Kunden einstellen muss \autocite{oleff2022proaktives, davis2006communication, gamil2017identification}.

Außerdem können Kunden Anforderungen aus Erwartungen haben, die unrealistisch und nicht für den Consultant umsetzbar sind \autocite{bjarnason2017role}.
\end{document}

