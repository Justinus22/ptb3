\documentclass[../main.tex]{subfiles}
\begin{document}

Anforderungen sind ausschlaggebend für das Verständnis eines Consultants über das Ziel eines Projekts und damit ein relevanter Aspekt der Kundenkommunikation \autocite{davis2006communication}.
Das eindeutige Identifizieren und Artikulieren der Anforderungen ist dabei ein Schlüsselpunkt und kann nur gelingen, wenn die Kenntnis darüber bei der Kundenseite vorhanden ist \autocite{gamil2017identification, chakrabarti2004identification, salado2021systems}.

Allein aus Unkenntnis können Anforderungen initial unverständlich formuliert oder nicht den eigentlichen Bedürfnissen entsprechend vorliegen \autocite{bjarnason2017role}.
Außerdem werden Anforderungen im Prozess häufig geändert, was ein unvermeidlicher und für den Projekterfolg notwendiger Schritt der Anpassung ist, der allerdings auch die Herausforderung mit sich bringt, dass Kunde und Consultant ihren Kenntnisstand immer wieder erneuern müssen \autocite{oleff2022proaktives, davis2006communication, gamil2017identification}.

Ein häufiges Problem beim Informationsaustausch ist außerdem die missverständliche oder unzureichende Zielsetzung.
Sind Ziele nicht deutlich identifiziert und kommuniziert, sind auch die Anforderungen an ein Projekt schwer festzustellen und die Ergebnisse später nur schwer zu evaluieren, da keine eindeutige Auslegung ermöglicht wird.
\autocite{cothran2005developing,appelbaum2005critical}

Des Weiteren können Kunden Anforderungen aus Erwartungen haben, die unrealistisch und nicht für den Consultant umsetzbar sind \autocite{bjarnason2017role}.
\end{document}