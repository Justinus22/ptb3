\documentclass[../main.tex]{subfiles}
\begin{document}
Verständnis ist das, was im Idealfall bei gelungener Auswahl von Information und Ausdruck resultiert und Wissen erzeugt.
Typisch sind jedoch Missverständnisse, das heißt, dass mindestens einer Partei Verständnis fehlt, weil eine Seite etwas nicht weiß oder nicht kann, was relevant für die Kommunikation gewesen wäre.
Die Kommunikation basiert dann auf falschen Annahmen. 
\\
Zur Verdeutlichung seien noch einmal folgende Beispiele genannt. 

Der Consultant benötigt zunächst einmal eine Grundkenntnis der Branche des Kunden, die er ggf. aber nicht hat, wovon der Kunde jedoch ausgeht \autocite{appelbaum2005critical,davis2006communication,gamil2017identification,sutter2019consultants}.
Der Kunde wiederum besitzt vielleicht keine exakte Vorstellung seiner eigenen Bedürfnisse und kann diese deshalb auch nicht transportieren, was dem Consultant eventuell entgeht \autocite{davis2006communication}.

Selbst wenn beide Parteien glauben, klare Anforderungen formuliert zu haben, kann die unterschiedliche Denkweise oder verschiedene Perspektiven von Kunde und Consultant zu Missverständnissen führen und die Seiten verkennen einander \autocite{sutter2019consultants}.
Auch Sprachbarrieren oder Ungenauigkeiten im Gebrauch natürlicher Sprache können hinderlich sein \autocite{gamil2017identification,sayer2013misunderstanding}.

\end{document}