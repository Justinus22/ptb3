\documentclass[../main.tex]{subfiles}
\begin{document}

Neben dem, was inhaltlich kommuniziert wird und wie es verstanden wird, sind auch die Wege, die Art und Weise und die Sprache, also der Ausdruck der Kommunikation mitunter problematisch. \autocite{cornelissen2023corporate}.

Dazu gehören die Kommunikationssysteme zwischen Kunden und Consultants. Probleme treten auf, wenn ungeeignete Plattformen zum direkten Austausch verwendet werden oder Plattformen zum Festhalten von Absprachen und Anforderungen fehlen.
\autocite{gamil2017identification,bjarnason2017role}

Weitere Probleme durch die Ausdrucksform in der Kommunikation können Gefühle des Einzelnen und zwischenmenschliche Empfindlichkeiten darstellen.
Dazu gehören z. B. Mangel an Respekt, Misstrauen oder Demotivation, die sich u. a. in Tonlage und Wortauswahl niederschlagen können. 
\autocite{mackenzie19597,appelbaum2005critical,bjarnason2017role}

Des Weiteren kann ein missverständlicher Ausdruck zu Herausforderungen führen, wenn z. B. Consultants den Kunden zu viel versprechen, zu hohe Erwartungen wecken und diese später nicht erfüllen können.
Dies begünstigt später oft zwischenmenschliche Probleme.
\autocite{mackenzie19597,appelbaum2005critical}

Zudem kann mangelndes Feedback beidseitig herausfordernd sein, da ohne Rückmeldung nicht bewertet kann, ob Anforderungen erfüllt werden oder nicht. 
\autocite{bano2014aligning,reihlen2010knowledge}

Eine nicht regelmäßige Kommunikation und das Fehlen eines klaren Kommunikationsplans kann ebenso herausfordernd sein.
\autocite{gamil2017identification,bjarnason2017role}
\end{document}