\documentclass[../main.tex]{subfiles}
\begin{document}

Neben dem was kommuniziert wird und dem wie es verstanden wird, sind auch die Bedingungen, unter denen die Kommunikation statt findet wesentlich \autocite{cornelissen2023corporate}.

Dazu gehören die Kommunikationssysteme zwischen Kunden und Consultants.
Das heißt, dass eine geeignete Plattform zum direkten Austausch sowie eine Plattform zum festhalten von Absprachen und Anforderungen wichtig ist.
\autocite{gamil2017identification,bjarnason2017role}

Außerdem sind zwischenmenschliche Probleme immer wieder herausfordernd.
Dazu gehören der Respekt, das Vertrauen und die Motivation, welche von beiden Seiten mitgebracht werden müssen.
\autocite{mackenzie19597,appelbaum2005critical,bjarnason2017role}

Des weiteren führt es zu Herausforderungen, wenn Consultant dem Kunden zu viel versprechen und dadurch zu hohe Erwartungen setzten.
Dies begünstig später oft die zwischenmenschlichen Probleme.
\autocite{mackenzie19597,appelbaum2005critical}

Eine Bedingungen unter der oft Anforderungen nicht klar werden sind Ziele.
Sind diese nicht deutlich identifiziert und kommuniziert, sind auch die Anforderungen schwer festzustellen und später zu interpretieren, da keine eindeutige Auslegung anhand der Ziele ermöglicht wird.
\autocite{cothran2005developing,appelbaum2005critical}

Zudem kann mangelndes Feedback des Kunden herausfordernd sein, da ohne dieses Feedback der Consultant nicht bewerten kann, ob er seine Arbeit den Kundenanforderungen entsprechend macht.
\autocite{bano2014aligning,reihlen2010knowledge}

Nicht regelmäßige Kommunikation und das Fehlen eines klaren Kommunikationsplans kann auch herausfordernd sein.
\autocite{gamil2017identification,bjarnason2017role}
\end{document}