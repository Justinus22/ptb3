\documentclass[
	12pt, %Schriftgröße
	a4paper,
	bibliography=totoc, %Inhaltsverzeichniseinträge f+r Quellen
	numbers=noenddot, %Entfernt Punkt hinter Gliederungsnummern
	ngerman, %Sprachpaket
	headsepline, %Headertrennlinie
	%footsepline, %Footertrennlinie
	oneside %einseitiges Druckformat %%% Unterdrücken der leeren Seite nach Titelblatt
	]{scrbook} %Dokumentenklasse (Koma-Script)

\usepackage{microtype}
\usepackage[T1]{fontenc}
\usepackage{float}
\usepackage[utf8]{inputenc}
\usepackage[ngerman]{babel}
\usepackage{url}
\usepackage{graphicx} %Bilder einfügen
%\usepackage{hyphenat}
%\usepackage{pdfpages} %PDF einfügen
\usepackage[a4paper, margin=1in]{geometry}
\usepackage[right]{eurosym} %Euro-Zeichen
\usepackage{amssymb}
\usepackage{setspace} % Zeilenabstand
\usepackage[ 
   colorlinks,        % Links ohne Umrandungen in zu wählender Farbe 
   linkcolor=black,   % Farbe interner Verweise 
   filecolor=black,   % Farbe externer Verweise 
   citecolor=black,   % Farbe von Zitaten 
   urlcolor=blue	  % Farbe von Links
   ]{hyperref} %Verlinkungen
\usepackage[figure]{hypcap}
\usepackage[ngerman]{translator}
\usepackage[acronym, nonumberlist, toc, translate=babel]{glossaries} %% Glossar mit Akronymen laden und mit babel auf deutsch anzeigen

\usepackage{csquotes}
\usepackage[]{bookmark}
\usepackage{scrhack}
      
\usepackage{listings,xcolor} %Codeanzeige
\usepackage[normalem]{ulem}
\useunder{\uline}{\ul}{}

\usepackage{array,longtable}
\usepackage{xltabular}
\usepackage{chngcntr}
\counterwithout{figure}{chapter}
\counterwithout{table}{chapter}

\usepackage{subfiles}

\usepackage{filecontents}
\usepackage{pgfplotstable}

\usepackage[
backend=biber,
style=apa,
]{biblatex}
\addbibresource{lit.bib}

\definecolor{dkgreen}{rgb}{0,.6,0}
\definecolor{dkblue}{rgb}{0,0,.6}
\definecolor{dkyellow}{cmyk}{0,0,.8,.3}

\lstset{
    numbers=left, 
    numberstyle=\tiny, 
    numbersep=5pt,
    breaklines=true,
    frame=single,
    language=sh,
    basicstyle=\ttfamily\fontsize{10}{12}\selectfont,
    keywordstyle    = \color{dkblue},
    stringstyle     = \color{red},
    identifierstyle = \color{dkgreen},
    commentstyle    = \color{gray},
    emph            =[1]{php},
    emphstyle       =[1]\color{black},
    emph            =[2]{if,and,or,else},
    emphstyle       =[2]\color{dkyellow}
} 

%%%%%%%%%%%%%%%%%%%%%%%%%%%%%%%%%%%%%%%%%%%%%%%%%%%%%
%%%%%%%%%%% Sonderformatierung
%%%%%%%%%%%%%%%%%%%%%%%%%%%%%%%%%%%%%%%%%%%%%%%%%%%%%

% Seitenabstände definieren
\geometry{verbose,tmargin=3cm,bmargin=2cm,lmargin=3cm,rmargin=3cm} 

% Hurenkinder und Schusterjungen verhindern (Ja, das heißt wirklich so!!!)
\clubpenalty= 10000
\widowpenalty= 10000
\displaywidowpenalty= 10000 

%% Bei Referenzen im Text wird jetzt bei allen Ebenen "Kapitel" vorgestellt, z.b. Kapitel 2, Kapitel 2.2, Kapitel 6.3.2
\addto\extrasngerman{
    \def\sectionautorefname{Kapitel}%
    \def\subsectionautorefname{Kapitel}%
    \def\subsubsectionautorefname{Kapitel}%
    }

% Vertikaler Abstand zwischen Ende Textblock - Ende Fußzeile --> Abstand der Seitenzahl von Rand erhöhen 
\setlength{\footskip}{10mm}

% Abstand vor/nach Überschriften verändern

\RedeclareSectionCommand[%
    beforeskip=0.5\baselineskip,
    afterskip=0.5\baselineskip,
]{chapter}

\RedeclareSectionCommand[%
    beforeskip=0.5\baselineskip,
    afterskip=0.5\baselineskip,
]{section}

\RedeclareSectionCommand[%
    beforeskip=0.1\baselineskip,
    afterskip=0.1\baselineskip,
]{subsection}

\RedeclareSectionCommand[%
    beforeskip=0.01\baselineskip,
    %%afterskip=0.2\baselineskip
]{paragraph}

\setlength{\abovecaptionskip}{4pt}  % 1pc=12pt 
\setlength{\belowcaptionskip}{0pt}
%\setlength{\textfloatsep}{4pt}
\setlength{\intextsep}{1pc}

%% Verkleinerung der Textgröße unter Abbildungen
\addtokomafont{caption}{\small}



% Den Punkt am Ende der Glossar-Einträge deaktivieren
\renewcommand*{\glspostdescription}{}

%Glossar-Befehle anschalten
\makenoidxglossaries
\loadglsentries{glossar.tex}

% sorgt dafür, dass bei Leerzeile die Einrückung verhindert und stattdessen eine Leerzeile eingefügt wird % erspart große Sprünge und erhöht die Lesbarkeit im LaTeX-Text 
\KOMAoptions{parskip=full*}

% ändert Titelschriftart in Serifen-Normalschriftart
\addtokomafont{disposition}{\rmfamily} 




%%%%%%%%%%%%%%%%%%%%%%%%%%%%%%%%%%%%%%%%%%%%%%%%%%%%%
\newcommand{\studentName}{Justin Becker}
\newcommand{\matrnr}{77204368351}
\newcommand{\type}{Praxistransferbericht}
\newcommand{\topic}{Herausforderungen in der Kundenkommunikation im Analytics Consulting}
\newcommand{\studiengangh}{Informatik}
\newcommand{\fachbereich}{FB2: Duales Studium --- Technik}
\newcommand{\studiengang}{Informatik}
\newcommand{\company}{SAP SE}
\newcommand{\betreuerHS}{Carl Dolling}
\newcommand{\betreuerUnt}{Jenny Peter}
\newcommand{\jahrgang}{2023}
%%%%%%%%%%%%%%%%%%%%%%%%%%%%%%%%%%%%%%%%%%%%%%%%%%%%%>>>>>>>


\newcommand{\quickwordcount}[1]{%
  \immediate\write18{texcount -1 -sum -merge -q -relaxed #1.tex > #1-words.sum }%
  \input{#1-words.sum}%
}

\DeclareCiteCommand{\@citelinktext}
  {}%
  {}
  {}
  {\printtext[bibhyperref]{\usebibmacro{postnote}}}

\makeatletter
\DeclareCiteCommand{\@citelinktext}
{}%
{}
{}
{\printtext[bibhyperref]{\usebibmacro{postnote}}}

\newcommand{\citelinktext}[2]{%
    \@citelinktext[#2]{#1}%
}
\makeatother

\newcounter{rowcntr}[table]
\renewcommand{\therowcntr}{\arabic{rowcntr}.}

\newcolumntype{R}[1]{>{\raggedleft\let\newline\\\arraybackslash\hspace{0pt}}m{#1}}
% A new columntype to apply automatic stepping
\newcolumntype{N}{>{\refstepcounter{rowcntr}\therowcntr}R{0.6cm}}

% Reset the rowcntr counter at each new tabular
\AtBeginEnvironment{tabular}{\setcounter{rowcntr}{0}}


\pgfplotsset{compat=newest}
\usepgfplotslibrary{statistics}
\usetikzlibrary{plotmarks}

\makeatletter
\pgfplotsset{
    boxplot/hide outliers/.code={
        \def\pgfplotsplothandlerboxplot@outlier{}%
    }
}
\makeatother


\newenvironment*{dontcount}{}{}
%TC:envir dontcount [ignore] displaymath

%%%%%%%%%%%%%%%%%%%%%%%%%%%%%%%%%%%%%%%%%%%%%%%%%%%%%%%%%%%%%%

\begin{document}

% falsche Default-Silbentrennung überschreiben
%\include{hyphenation}

% keine Silbentrennung sondern space zwischen den Wörtern nutzen, wie Block Absatz in word 
\spaceskip=0.3em plus 4.5em minus 0.3em

%%%%%%%%%%%%%%%%%%%%%%%%%%%%%%%%%%%%%%%%%%%%%%%%%%%%%>>>>>>>
%%%%%%%%%%% Titelblatt

%% Anordnung und Aussehen von Titel und Untertitel

\subject{\type}

\title{
\normalfont\endgraf\rule{\textwidth}{.4pt}
\begingroup
	\centering
	\linespread{1.5}
	\huge\topic%
\endgroup
\endgraf\rule{\textwidth}{.4pt}
}
 
%%Eigentlich nicht besonders schön, aber Koma erlaubt die Anordnung eines weiteren Felder (hier: Fachbereich) nicht
\date{\vspace{-2cm}\normalsize vorgelegt am \today \vspace{1cm}}
%% \date muss leer angegeben werden, um die Default-Datumsanzeige

\publishers{
	\begin{tabular}{l l}
	\textbf{\normalsize{Name:}} & \normalsize{\studentName}  \tabularnewline%
    \textbf{\normalsize{Matrikelnummer:}} & \normalsize{\matrnr}  \tabularnewline%
	\textbf{\normalsize{Ausbildungsbetrieb:}} & \normalsize{\company}  \tabularnewline%
    \textbf{\normalsize{Fachbereich:}} & \normalsize{\fachbereich} \tabularnewline%
    \textbf{\normalsize{Studienjahrgang:}} & \normalsize{\jahrgang} \tabularnewline%
	\textbf{\normalsize{Studiengang:}} & \normalsize{\studiengang}  \tabularnewline%
	\textbf{\normalsize{Betreuer Unternehmen:}} & \normalsize{\betreuerUnt} \tabularnewline%
	\textbf{\normalsize{Betreuer Hochschule:}} & \normalsize{\betreuerHS} \tabularnewline%
	\textbf{\normalsize{Wortanzahl:}} & \normalsize{\quickwordcount{main}}
	\end{tabular}
}

\titlehead{\begin{center}
    \includegraphics[scale=0.7]{bilder/header_logo.PNG}
    \end{center}
    }

\maketitle

\onehalfspacing

\begin{dontcount}
\textbf{Sperrvermerk}

Der vorliegende Bericht enthält vertrauliche Daten des Unternehmens SAP SE.
Auf Wunsch des Unternehmens SAP SE ist der vorliegende Bericht für die öffentliche Nutzung zu sperren.
Veröffentlichung, Vervielfältigung und Einsichtnahme sind ohne ausdrückliche Genehmigung des Unternehmen SAP SE, in 14469 Potsdam und des Verfassers Justin Becker nicht gestattet.
Der Bericht ist nur den Gutachtern und den Mitgliedern des Prüfungsausschusses zugänglich zu machen.

\vspace{20mm}

\begin{tabular}{lp{2em}l} 
    \hspace{4cm}   && \hspace{4cm} \\\cline{1-1}\cline{3-3} 
    Ort, Datum     && \studentName{}
\end{tabular}

\vspace{\fill}
\normalsize{Von der betrieblichen Betreuung zur Kenntnis genommen:}
\vspace*{20mm}

\begin{tabular}{lp{2em}l} 
    \hspace{4cm}   && \hspace{4cm} \\\cline{1-1}\cline{3-3} 
    Ort, Datum     && \betreuerUnt
\end{tabular}

\end{dontcount}

%%%%%%%%%%%%%%%%%%%%%%%%%%%%%%%%%%%%%%%%%%%%%%%%%%%%%%%%%%%%%%%%%%%%%%%%%%%%%%%%%%%%%%%%%%%%%%%%%%%%%%%%%%%%%%%%%%%%%%%%%%%
%%%%%%%%%%% START
%%%%%%%%%%%%%%%%%%%%%%%%%%%%%%%%%%%%%%%%%%%%%%%%%%%%%%%%%%%%%%%%%%%%%%%%%%%%%%%%%%%%%%%%%%%%%%%%%%%%%%%%%%%%%%%%%%%%%%%%%%%

%%%%%%%%%%%%%%%%%%%%%%%%%%%%%%%%%%%%%%%%%%%%%%%%%%%%%
%%%%%%%%%%% Abstract
\chapter*{Zusammenfassung}
\addcontentsline{toc}{chapter}{Zusammenfassung}
\subfile{chapter/abstract.tex}

\pagenumbering{Roman} % römische Seitenzahlen

%%%%%%%%%%%%%%%%%%%%%%%%%%%%%%%%%%%%%%%%%%%%%%%%%%%%%
%%%%%%%%%%% Inhaltsverzeichnis, Tabellen, Abbildungen, etc.
\newpage

\tableofcontents{}
\addcontentsline{toc}{chapter}{Inhaltsverzeichnis}

\listoffigures
\addcontentsline{toc}{chapter}{Abbildungsverzeichnis}

\begin{dontcount}
\section*{Hinweis}

Aus Gründen der besseren Lesbarkeit wird im Text verallgemeinernd die männliche Form verwendet.
Diese Formulierungen umfassen gleichermaßen weibliche und männliche Personen.
\clearpage

\end{dontcount}
%Tabellenverzeichnis, falls nötig: 

%\listoftables
%\addcontentsline{toc}{chapter}{Tabellenverzeichnis}


\printnoidxglossary{}
\printacronyms{}

\clearpage

%% arabische Seitenzahlen
\pagenumbering{arabic}

%%%%%%%%%%%%%%%%%%%%%%%%%%%%%%%%%%%%%%%%%%%%%%%%%%%%%
%%%%%%%%%%% Einführung

\chapter{Einleitung}\label{ch:einleitung}
\subfile{chapter/einleitung.tex}

\chapter{Theoretische Betrachtung}\label{ch:theroy}
\subfile{chapter/therory.tex}


\section{Herausforderungen}\label{sec:herausforderungen}
\subfile{section/herausforderungen.tex}

\subsection{Anforderungen}\label{subsec:anforderungen}
\subfile{subsection/anforderungen.tex}


\subsection{Ausdruck}\label{subsec:umgebung}
\subfile{subsection/umgebung.tex}

\subsection{Verständnis}\label{subsec:verstaendins}
\subfile{subsection/verständnis.tex}

% \textcolor{red}{
% \section{Besonderheiten im Analytics Consulting}\label{sec:besonderheiten}
% \subfile{section/besonderheiten.tex}
% }
\chapter{Umfrage zu Herausforderungen}\label{ch:umfrage}
\subfile{chapter/umfrage.tex}

\section{Aufbau}\label{sec:aufbau}
\subfile{section/aufbau.tex}

\section{Auswertung}\label{sec:auswertung}
\subfile{section/auswertung.tex}

\section{Bewertung}\label{sec:bewertung}
\subfile{section/bewertung.tex}


\chapter{Fazit}\label{ch:fazit}
\subfile{chapter/fazit.tex}

%%%%%%%%%%%%%%%%%%%%%%%%%%
% Quellen
%%%%%%%%%%%%%%%%%%%%%%%%%
\spaceskip=0.3em plus 0.5em minus 0.2em
\printbibliography{}


\chapter*{Ehrenwörtliche Erklärung}
\addcontentsline{toc}{chapter}{Ehrenwörtliche Erklärung}
\subfile{chapter/ehrenwörtliche_erklärung.tex}

\end{document}

